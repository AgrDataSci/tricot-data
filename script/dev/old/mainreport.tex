% Options for packages loaded elsewhere
\PassOptionsToPackage{unicode}{hyperref}
\PassOptionsToPackage{hyphens}{url}
%
\documentclass[
]{article}
\usepackage{amsmath,amssymb}
\usepackage{iftex}
\ifPDFTeX
  \usepackage[T1]{fontenc}
  \usepackage[utf8]{inputenc}
  \usepackage{textcomp} % provide euro and other symbols
\else % if luatex or xetex
  \usepackage{unicode-math} % this also loads fontspec
  \defaultfontfeatures{Scale=MatchLowercase}
  \defaultfontfeatures[\rmfamily]{Ligatures=TeX,Scale=1}
\fi
\usepackage{lmodern}
\ifPDFTeX\else
  % xetex/luatex font selection
\fi
% Use upquote if available, for straight quotes in verbatim environments
\IfFileExists{upquote.sty}{\usepackage{upquote}}{}
\IfFileExists{microtype.sty}{% use microtype if available
  \usepackage[]{microtype}
  \UseMicrotypeSet[protrusion]{basicmath} % disable protrusion for tt fonts
}{}
\makeatletter
\@ifundefined{KOMAClassName}{% if non-KOMA class
  \IfFileExists{parskip.sty}{%
    \usepackage{parskip}
  }{% else
    \setlength{\parindent}{0pt}
    \setlength{\parskip}{6pt plus 2pt minus 1pt}}
}{% if KOMA class
  \KOMAoptions{parskip=half}}
\makeatother
\usepackage{xcolor}
\usepackage[margin=1in]{geometry}
\usepackage{graphicx}
\makeatletter
\def\maxwidth{\ifdim\Gin@nat@width>\linewidth\linewidth\else\Gin@nat@width\fi}
\def\maxheight{\ifdim\Gin@nat@height>\textheight\textheight\else\Gin@nat@height\fi}
\makeatother
% Scale images if necessary, so that they will not overflow the page
% margins by default, and it is still possible to overwrite the defaults
% using explicit options in \includegraphics[width, height, ...]{}
\setkeys{Gin}{width=\maxwidth,height=\maxheight,keepaspectratio}
% Set default figure placement to htbp
\makeatletter
\def\fps@figure{htbp}
\makeatother
\setlength{\emergencystretch}{3em} % prevent overfull lines
\providecommand{\tightlist}{%
  \setlength{\itemsep}{0pt}\setlength{\parskip}{0pt}}
\setcounter{secnumdepth}{-\maxdimen} % remove section numbering
% definitions for citeproc citations
\NewDocumentCommand\citeproctext{}{}
\NewDocumentCommand\citeproc{mm}{%
  \begingroup\def\citeproctext{#2}\cite{#1}\endgroup}
\makeatletter
 % allow citations to break across lines
 \let\@cite@ofmt\@firstofone
 % avoid brackets around text for \cite:
 \def\@biblabel#1{}
 \def\@cite#1#2{{#1\if@tempswa , #2\fi}}
\makeatother
\newlength{\cslhangindent}
\setlength{\cslhangindent}{1.5em}
\newlength{\csllabelwidth}
\setlength{\csllabelwidth}{3em}
\newenvironment{CSLReferences}[2] % #1 hanging-indent, #2 entry-spacing
 {\begin{list}{}{%
  \setlength{\itemindent}{0pt}
  \setlength{\leftmargin}{0pt}
  \setlength{\parsep}{0pt}
  % turn on hanging indent if param 1 is 1
  \ifodd #1
   \setlength{\leftmargin}{\cslhangindent}
   \setlength{\itemindent}{-1\cslhangindent}
  \fi
  % set entry spacing
  \setlength{\itemsep}{#2\baselineskip}}}
 {\end{list}}
\usepackage{calc}
\newcommand{\CSLBlock}[1]{\hfill\break\parbox[t]{\linewidth}{\strut\ignorespaces#1\strut}}
\newcommand{\CSLLeftMargin}[1]{\parbox[t]{\csllabelwidth}{\strut#1\strut}}
\newcommand{\CSLRightInline}[1]{\parbox[t]{\linewidth - \csllabelwidth}{\strut#1\strut}}
\newcommand{\CSLIndent}[1]{\hspace{\cslhangindent}#1}
\ifLuaTeX
  \usepackage{selnolig}  % disable illegal ligatures
\fi
\usepackage{bookmark}
\IfFileExists{xurl.sty}{\usepackage{xurl}}{} % add URL line breaks if available
\urlstyle{same}
\hypersetup{
  pdftitle={Common bean on-farm trials in Tanzania},
  hidelinks,
  pdfcreator={LaTeX via pandoc}}

\title{Common bean on-farm trials in Tanzania}
\author{}
\date{\vspace{-2.5em}27 November, 2024}

\begin{document}
\maketitle

{
\setcounter{tocdepth}{2}
\tableofcontents
}
\section{Introduction}\label{introduction}

This report presents the results from your tricot experiment, offering
insights into the performance of the technologies tested in the
experiment. Designed as a standard automated report, it highlights the
most critical outputs from your tricot experiment, leveraging
farmer-generated data to ensure relevance and reliability. By reviewing
this report, you will be able to: (i) identify which technology
outperforms others across diverse conditions and farmer contexts, and
(ii) make informed decisions about advancing a technology to the next
stage of your product development pipeline (e.g., breeding program,
market assessment). This report reflects the tricot approach's strength
in providing actionable, farmer-centered insights that support scalable,
demand-driven innovation.

\section{Methods}\label{methods}

We applied the Plackett-Luce (PL) model, originally proposed
independently by Luce (1959) (\emph{1}) and Plackett (1975) (\emph{2}),
to analyze ranking data. This model is implemented in R through the
PlackettLuce package (\emph{3}). The Plackett-Luce model estimates the
relative importance or probability of outperforming among different
technologies, adhering to Luce's Choice Axiom (\emph{1}). This axiom
asserts that the probability of one item (e.g., variety) outperforming
another is independent of the presence or absence of other items in the
set, enabling robust and context-independent comparisons.

Equation {[}1{]}

\[P (i \succ j) = \frac{p_i}{p_i + p_j}\]

where \(p_i\) is a positive real-valued score assigned to individual
\(i\). The comparison \(i \succ j\) can be read as ``\(i\) is preferred
over \(j\)''.

The PL model determines the values of positive-valued parameters
\(\alpha_i\) (\emph{worth}) associated with each item \(i\). These
parameters \(\alpha\) are related to the probability (\(P\)) that item
\(i\) outperforms all other \(n\) items in the set.

Your tricot experiment used rankings of three technologies
\((i \succ j \succ k)\), which have the following probability of
occurring according to the PL model:

Equation {[}2{]}

\[P(i \succ j \succ k) = P(i \succ {j,k}) \cdot P(j \succ k)\]

he correlation between `overall preference' and other traits is
estimated using the Kendall tau coefficient (\emph{4}). This
coefficient, a non-parametric equivalent to the Pearson correlation, is
specifically designed for ranking data. It ranges from -1 to 1, where 1
indicates perfect positive correlation and -1 indicates perfect negative
correlation. This method is particularly useful for identifying the
drivers behind participants' choices and for prioritizing traits to test
in the next stage of tricot experiments, ensuring that future efforts
are aligned with key farmer preferences and priorities.

This report is generated in R (\emph{5}) using the R packages `knitr'
(\emph{6}) and `rmarkdown' (\emph{7}). Organizing the data relies on
packages `ClimMobTools' (\emph{8}), `gosset' (\emph{9}), `gtools'
(\emph{10}), `jsonlite' (\emph{11}), `partykit' (\emph{12}),
`psychotools' (\emph{13}) and `qvcalc' (\emph{14}). Summaries and data
visualization are supported by packages `igraph' (\emph{15}), `ggparty'
(\emph{16}), `ggplot2' (\emph{17}), `ggrepel' (\emph{18}), `gridExtra'
(\emph{19}) `leaflet' (\emph{20}), `multcompView' (\emph{21}),
`patchwork' (\emph{22}), `png' (\emph{23}), `plotrix' (\emph{24}) and
`pls' (\emph{25}). The decentralized experimental approach behind
ClimMob is introduced by van Etten et al.~(2019) (\emph{26}). To cite
ClimMob itself, mention van Etten et al.~(2020) (\emph{27}). The
workflow used to produce this report is documented by de Sousa et
al.~(2022) (\emph{28}).

\pagebreak

\section{Trial overview}\label{trial-overview}

The map below (Fig. 2) shows the distribution of the plots in this
experiment.

\begin{figure}
\includegraphics[width=1\linewidth]{../output/bean-eastafrica/Rplot} \caption{Fig. 1. Distribution of trial plots.}\label{fig:map}
\end{figure}

Figure 2 provides a graphical representation of the experimental design,
illustrating whether all varieties are interconnected within the
experiment network. A fully connected network indicates that all
varieties co-occurred at least once within an incomplete block of 3. The
network structure is derived from the rankings generated based on
overall preference, ensuring the visualization reflects the
relationships informed by participant preferences.

\begin{figure}
\includegraphics[width=50in]{../output/bean-eastafrica/trial-network} \caption{Figure 2. Experimental network representation of varieties tested in this experiment. Arrows indicate direct paths of wins and losses between each pair of variety indicating that they co-occur in at least one experimental block.}\label{fig:trial_network}
\end{figure}

\section{Summary of variety
performance}\label{summary-of-variety-performance}

Table 1 presents the results of an analysis of variance (ANOVA), which
tests the hypothesis that at least one variety in the set performs
significantly better than the others. Traits with a p-value \textless{}
0.05 suggest that the observed performance differences are not due to
random variation, indicating that at least one variety exhibits superior
performance.

\section{References}\label{references}

\phantomsection\label{refs}
\begin{CSLReferences}{0}{1}
\bibitem[\citeproctext]{ref-Luce1959}
\CSLLeftMargin{1. }%
\CSLRightInline{R. D. Luce, \emph{{Individual Choice Behavior}} (Courier
Corporation, 1959).}

\bibitem[\citeproctext]{ref-Plackett1975}
\CSLLeftMargin{2. }%
\CSLRightInline{R. L. Plackett,
\href{https://doi.org/10.2307/2346567}{{The Analysis of Permutations}}.
\emph{Journal of the Royal Statistical Society. Series C (Applied
Statistics)} \textbf{24}, 193--202 (1975).}

\bibitem[\citeproctext]{ref-Turner2020}
\CSLLeftMargin{3. }%
\CSLRightInline{H. L. Turner, J. van Etten, D. Firth, I. Kosmidis,
{Modelling rankings in R: the PlackettLuce package}. \emph{Computational
Statistics}, doi:
\href{https://doi.org/10.1007/s00180-020-00959-3}{10.1007/s00180-020-00959-3}
(2020).}

\bibitem[\citeproctext]{ref-Kendall1938}
\CSLLeftMargin{4. }%
\CSLRightInline{M. G. Kendall,
\href{https://doi.org/10.1093/biomet/30.1-2.81}{{A new measure of
ranking correlation}}. \emph{Biometrika} \textbf{30}, 81--93 (1938).}

\bibitem[\citeproctext]{ref-RCoreTeam}
\CSLLeftMargin{5. }%
\CSLRightInline{R Core Team, {R: A language and environment for
statistical computing. version 3.6.2.} (2019).
\url{https://r-project.org/}.}

\bibitem[\citeproctext]{ref-knitr}
\CSLLeftMargin{6. }%
\CSLRightInline{Y. Xie, \emph{{Dynamic Documents with {R} and knitr}}
(Chapman; Hall/CRC, Boca Raton, Florida, ed. 2nd, 2015;
\url{https://yihui.org/knitr/}).}

\bibitem[\citeproctext]{ref-rmarkdown}
\CSLLeftMargin{7. }%
\CSLRightInline{Y. Xie, J. J. Allaire, G. Grolemund, \emph{{R Markdown:
The Definitive Guide}} (Chapman; Hall/CRC, Boca Raton, Florida, 2018;
\url{https://bookdown.org/yihui/rmarkdown}).}

\bibitem[\citeproctext]{ref-climmobtools}
\CSLLeftMargin{8. }%
\CSLRightInline{K. de Sousa, J. van Etten, {{ClimMobTools}: API Client
for the {'ClimMob'} platform} (2022).
\url{https://CRAN.R-project.org/package=ClimMobTools}.}

\bibitem[\citeproctext]{ref-gosset}
\CSLLeftMargin{9. }%
\CSLRightInline{K. de Sousa, D. Brown, J. Steinke, J. van Etten,
{gosset: An R Package for Analysis and Synthesis of Ranking Data in
Agricultural Experimentation}. \emph{{SSRN} Electronic Journal}, doi:
\href{https://doi.org/10.2139/ssrn.4236267}{10.2139/ssrn.4236267}
(2022).}

\bibitem[\citeproctext]{ref-gtools}
\CSLLeftMargin{10. }%
\CSLRightInline{G. R. Warnes, B. Bolker, T. Lumley, {{gtools}: Various R
Programming Tools} (2018).
\url{https://CRAN.R-project.org/package=gtools}.}

\bibitem[\citeproctext]{ref-jsonlite}
\CSLLeftMargin{11. }%
\CSLRightInline{J. Ooms, \href{https://arxiv.org/abs/1403.2805}{{The
jsonlite Package: A Practical and Consistent Mapping Between JSON Data
and R Objects}}. \emph{arXiv:1403.2805 {[}stat.CO{]}} (2014).}

\bibitem[\citeproctext]{ref-partykit}
\CSLLeftMargin{12. }%
\CSLRightInline{A. Zeileis, T. Hothorn, K. Hornik,
\href{https://doi.org/10.1198/106186008X319331}{{Model-Based Recursive
Partitioning}}. \emph{Journal of Computational and Graphical Statistics}
\textbf{17}, 492--514 (2008).}

\bibitem[\citeproctext]{ref-psychotools}
\CSLLeftMargin{13. }%
\CSLRightInline{J. Kopf, C. Strobl, A. Zeileis,
\href{https://doi.org/10.1177/0013164414529792}{{Anchor Selection
Strategies for {DIF} Analysis: Review, Assessment, and New Approaches}}.
\emph{Educational and Psychological Measurement} \textbf{75}, 22--56
(2015).}

\bibitem[\citeproctext]{ref-qvcalc}
\CSLLeftMargin{14. }%
\CSLRightInline{D. Firth, {{qvcalc}: Quasi Variances for Factor Effects
in Statistical Models} (2020).
\url{https://CRAN.R-project.org/package=qvcalc}.}

\bibitem[\citeproctext]{ref-igraph}
\CSLLeftMargin{15. }%
\CSLRightInline{G. Csardi, T. Nepusz, \href{http://igraph.org}{The
igraph software package for complex network research}.
\emph{InterJournal} \textbf{Complex Systems}, 1695 (2006).}

\bibitem[\citeproctext]{ref-ggparty}
\CSLLeftMargin{16. }%
\CSLRightInline{M. Borkovec, N. Madin, \emph{{{ggparty}: 'ggplot'
Visualizations for the 'partykit' Package}} (2019;
\url{https://CRAN.R-project.org/package=ggparty}).}

\bibitem[\citeproctext]{ref-ggplot2}
\CSLLeftMargin{17. }%
\CSLRightInline{H. Wickham, \emph{{{ggplot2}: Elegant Graphics for Data
Analysis}} (Springer-Verlag New York, 2016;
\url{https://ggplot2.tidyverse.org}).}

\bibitem[\citeproctext]{ref-ggrepel}
\CSLLeftMargin{18. }%
\CSLRightInline{K. Slowikowski, \emph{{{ggrepel}: Automatically Position
Non-Overlapping Text Labels with 'ggplot2'}} (2020;
\url{https://CRAN.R-project.org/package=ggrepel}).}

\bibitem[\citeproctext]{ref-gridExtra}
\CSLLeftMargin{19. }%
\CSLRightInline{B. Auguie, \emph{{gridExtra: Miscellaneous Functions for
"Grid" Graphics}} (2017;
\url{https://CRAN.R-project.org/package=gridExtra}).}

\bibitem[\citeproctext]{ref-leaflet}
\CSLLeftMargin{20. }%
\CSLRightInline{J. Cheng, B. Karambelkar, Y. Xie, \emph{{leaflet: Create
Interactive Web Maps with the JavaScript 'Leaflet' Library}} (2019;
\url{https://CRAN.R-project.org/package=leaflet}).}

\bibitem[\citeproctext]{ref-multcompView}
\CSLLeftMargin{21. }%
\CSLRightInline{S. Graves, H.-P. Piepho, L. S. with help from Sundar
Dorai-Raj, \emph{{{multcompView}: Visualizations of Paired Comparisons}}
(2019; \url{https://CRAN.R-project.org/package=multcompView}).}

\bibitem[\citeproctext]{ref-patchwork}
\CSLLeftMargin{22. }%
\CSLRightInline{T. L. Pedersen, \emph{{{patchwork}: The Composer of
Plots}} (2019; \url{https://CRAN.R-project.org/package=patchwork}).}

\bibitem[\citeproctext]{ref-png}
\CSLLeftMargin{23. }%
\CSLRightInline{S. Urbanek, \emph{{png: Read and write PNG images}}
(2013; \url{https://CRAN.R-project.org/package=png}).}

\bibitem[\citeproctext]{ref-plotrix}
\CSLLeftMargin{24. }%
\CSLRightInline{L. J, {Plotrix: a package in the red light district of
R}. \emph{R-News} \textbf{6}, 8--12 (2006).}

\bibitem[\citeproctext]{ref-pls}
\CSLLeftMargin{25. }%
\CSLRightInline{B.-H. Mevik, R. Wehrens, K. H. Liland, {{pls}: Partial
Least Squares and Principal Component Regression} (2019).
\url{https://CRAN.R-project.org/package=pls}.}

\bibitem[\citeproctext]{ref-vanEtten2019tricot}
\CSLLeftMargin{26. }%
\CSLRightInline{J. van Etten, E. Beza, L. Calderer, K. Van Duijvendijk,
C. Fadda, B. Fantahun, Y. G. Kidane, J. van de Gevel, A. Gupta, D. K.
Mengistu, D. Kiambi, P. N. Mathur, L. Mercado, S. Mittra, M. J. Mollel,
J. C. Rosas, J. Steinke, J. G. Suchini, K. S. Zimmerer,
\href{https://doi.org/10.1017/S0014479716000739}{{First experiences with
a novel farmer citizen science approach: crowdsourcing participatory
variety selection through on-farm triadic comparisons of technologies
(tricot)}}. \emph{Experimental Agriculture} \textbf{55}, 275--296
(2019).}

\bibitem[\citeproctext]{ref-climmob}
\CSLLeftMargin{27. }%
\CSLRightInline{J. van Etten, R. Manners, J. Steinke, E. Matthus, K. de
Sousa, \emph{{The tricot approach: Guide for large-scale participatory
experiments}} (Bioversity International; Bioversity International, Rome,
Italy, 2020; \url{https://hdl.handle.net/10568/109942}).}

\bibitem[\citeproctext]{ref-climmobanalysis}
\CSLLeftMargin{28. }%
\CSLRightInline{K. de Sousa, B. Madriz, A. Muller, J. van Etten,
{Workflow for automated analysis and report of decentralized
experimental data with the tricot approach}. doi:
\href{https://doi.org/10.5281/zenodo.4609907}{10.5281/zenodo.4609907}
(2022).}

\end{CSLReferences}

\end{document}
